\documentclass[aspectratio=169,hyperref={pdfpagelabels=false}]{beamer}
\setbeameroption{show notes}

\usepackage{helvet}
\usepackage[english]{babel}
\usepackage{pgfplots}
\usepackage{pgf}
\pgfplotsset{compat=newest}
\usepackage{booktabs}
\usepackage[T1]{fontenc}
\usepackage[utf8]{inputenc}
\usepackage{lipsum}
\usepackage{tcolorbox}
\usepackage{xcolor}
\usepackage{listings}
\usepackage[nopatch=footnote]{microtype}
\usepackage{float}
\usepackage{siunitx}
\usepackage{multicol}
%\usepackage{hyperref}
\usepackage{dsfont}
\usepackage{caption}
\usepackage{subcaption}

\input{settings}

\newcommand{\setcolor}[1]{\def\chosencolor{#1}}
\newcommand{\setdepartment}[1]{\def\department{#1}}

\usetheme{DTU}
\setbeamersize{text margin left=22mm}
\def\insertframetitle{}

\newcommand{\inserttitlepage}{
    \begin{frame}[plain, noframenumbering]{}

        \begin{center}
            \hspace{-3.6em}
            \includegraphics[width = 0.25\paperwidth]{logos/\targetcolourmodel/white.pdf}
        \end{center}

    \end{frame}

    \begin{frame}[plain]{}
        \color{white}\maketitle
    \end{frame}

    \setbeamercolor{background canvas}{bg = white}
}


\newcommand{\documenttype}{Master Thesis}
\newcommand{\thesistitle}{Clusters in SAE Latents}
\newcommand{\thesissubtitle}{Characterizing Sparse AutoEncoder latents for LLM interpretability}

\newcommand{\thesisauthor}{Albert Ramus Sidenius Garde} % Your name :) 
\newcommand{\studentnumber}{s183969}
\newcommand{\thedate}{\today} % For example "June, 2019"

\newcommand{\thesisdepartment}{DTU Compute}
\newcommand{\departmentdescriber}{Department of Applied Mathematics and Computer Science}
\newcommand{\addressI}{Richard Petersens Plads, Building 324}
\newcommand{\addressII}{2800 Kgs. Lyngby}
\newcommand{\departmentwebsite}{www.compute.dtu.dk}

\usepackage[utf8]{inputenc}
\usepackage{fullpage}

% Stop Vincents stupid indryk ved at sætte længden af dem til 0
\setlength\parindent{0pt}

% Selectively silence warnings
\usepackage{silence}
\WarningFilter{hyperref}{Token not allowed}

% Math notation
\usepackage{amsmath}
\usepackage{amssymb}
\usepackage{amsfonts}
\usepackage{amsthm}
\usepackage{mathrsfs}
\usepackage{bm} % Used for bold math

% Subscript before
\usepackage{tensor}

% Framing
\usepackage{tcolorbox}
\tcbuselibrary{theorems}

% Placement of figures
\usepackage{float}

% Figures
\usepackage{graphicx}
\usepackage{subfig}

% Pseudocode
\usepackage{algorithmic}

% Enumerations
\usepackage{enumitem}

% Potter proof!
\usepackage{stmaryrd}

% Units
\usepackage{siunitx}

% Todos
\usepackage{todonotes}

% Hyperref
\usepackage{hyperref}

% More complex arguments
\usepackage{xparse}

% Symbols
%% Arabic symbols
\usepackage{arabtex}
\setarab
\arabtrue

\newcommand{\arj}{\RL{j}}
\newcommand{\ark}{\RL{.k}}
\newcommand{\artea}{\RL{.s}}
\newcommand{\ardoublemu}{\RL{s}}
\newcommand{\arsmiley}{\RL{t}}
\newcommand{\arhandcuffs}{\RL{mh}}

%% Standard symbols
\newcommand{\e}{\mathrm{e}}
\newcommand{\im}{\mathrm{i}}
\newcommand{\diff}{\mathrm{d}}
\newcommand{\id}{\mathrm{id}}
\renewcommand{\R}{\mathbb{R}}
\newcommand{\C}{\mathbb{C}}
\newcommand{\N}{\mathbb{N}}
\newcommand{\Z}{\mathbb{Z}}
\newcommand{\F}{\mathbb{F}}
\newcommand{\M}{\mathbb{M}}
\newcommand{\xor}{\oplus}


% Theorem boxes
\newcommand{\ksjlcisavbff}{ksjlcisavbff}


\newtcbtheorem{theorem}{Theorem}%
{colframe=black,fonttitle=\bfseries,
separator sign={\ -}}{theorem}
\newtcbtheorem[use counter from=theorem]{lemma}{Lemma}%
{colframe=black,fonttitle=\bfseries,
separator sign={\ -}}{lemma}
\newtcbtheorem[use counter from=theorem]{corollary}{Corollary}%
{colframe=black,fonttitle=\bfseries,
separator sign={\ -}}{corollary}
\newtcbtheorem[use counter from=theorem]{definition}{Definition}%
{colframe=black,fonttitle=\bfseries,
separator sign={\ -}}{definition}
\newtcbtheorem[use counter from=theorem]{example}{Example}%
{colframe=black,fonttitle=\bfseries,
separator sign={\ -}}{example}
\newtcbtheorem[use counter from=theorem]{problem}{Problem}%
{colframe=black,fonttitle=\bfseries,
separator sign={\ -}}{problem}

%\newmdtheoremenv[nobreak=true]{theorem}{Theorem}
%\newmdtheoremenv[nobreak=true]{lemma}{Lemma}
%\newmdtheoremenv[nobreak=true]{corollary}{Corollary}
%\newmdtheoremenv[nobreak=true]{definition}{Definition}
%\newmdtheoremenv[nobreak=true]{problem}{Problem}
%\newmdtheoremenv{example}{Example}

% Symbol formatting
\renewcommand{\vec}[1]{\mathbf{#1}}
\newcommand{\mat}[1]{\mathbf{#1}}
\newcommand{\set}[1]{\mathcal{#1}}

% Vector span
\newcommand{\vspan}{\mathrm{span}}

% Column vectors
\makeatletter
\newcommand{\cvec}[1]{
    \left[\begin{array}{c}#1\cvecchecknextarg}
\newcommand{\cvecchecknextarg}{\@ifnextchar\bgroup{\cvecgobblenextarg}{\end{array}\right]}}
\newcommand{\cvecgobblenextarg}[1]{\\#1\@ifnextchar\bgroup{\cvecgobblenextarg}{\end{array}\right]}}
\makeatother

% Row vectors
\makeatletter
\newcommand{\rvec}[2]{
    \left[\begin{array}{*{#1}{c}}#2\rvecchecknextarg}
\newcommand{\rvecchecknextarg}{\@ifnextchar\bgroup{\rvecgobblenextarg}{\end{array}\right]}}
\newcommand{\rvecgobblenextarg}[1]{&#1\@ifnextchar\bgroup{\rvecgobblenextarg}{\end{array}\right]}}
\makeatother

% Vertical iff
\newcommand{\viff}{\Updownarrow}

% Absolute
\newcommand{\abs}[1]{\left\lvert#1\right\lvert}

% Norm
\newcommand{\norm}[1]{\left\lvert\left\lvert#1\right\lvert\right\lvert}

% Inner product
\newcommand{\iprod}[2]{\left\langle#1,#2\right\rangle}

% Complex conjugate
\newcommand{\conj}[1]{\overline{#1}}

% Transpose
\newcommand{\trans}[1]{#1^\top}

% Conjugate transpose (adjunct)
\newcommand{\adj}[1]{#1^*}

% Infinite sum
\newcommand{\infsum}[2][1]{\sum_{#2=#1}^\infty}

% Label specific row in align
\newcommand\numberthis[1]{\stepcounter{equation}\label{#1}\tag{\theequation}}

% Conditional probability
\newcommand\given[1][]{\:#1\vert\:}

% Contradiction
\newcommand{\contradiction}{\ensuremath{\lightning}}

% Closure
\newcommand{\closure}[1]{\overline{#1}}

% Matrix trace
\newcommand{\trace}{\mathrm{tr}}

% Map image
\DeclareMathOperator{\ima}{im}

% Kronecker product	
\newcommand{\kronprod}{\otimes}

% Direct sum
\newcommand{\dirsum}{\oplus}

% Element-wise product
\newcommand{\elemprod}{\circ}
 
% Matrix representation of linear operator in basis
\newcommand{\linopbasis}[3]{\tensor[_{#1}]{\left[#2\right]}{_{#3}}}

\newcommand{\powerset}[1]{\mathcal{P}\left(#1\right)}

\DeclareMathOperator*{\argmax}{arg\,max}
\DeclareMathOperator*{\argmin}{arg\,min}

% Remove numbers for headers
%\setcounter{secnumdepth}{-2}

% Image
\newcommand{\image}[3]{
    \begin{figure}[H]
        \centering
        \includegraphics[scale=#3]{#1} 
        \caption{#2}
    \end{figure}
} 

% Itemize characters
\renewcommand\labelitemii{\textbullet}
\renewcommand\labelitemiii{\textbullet}

\newcommand\matlist[2]{#1_1,#1_2,\dots,#1_{#2}}

% References a Theorem/Lemma/Corollary depending on the label (thr/lemma/cor).
% For example, if a label thr:hairy_ball exists, then \thref{hairy_ball} will evaluate to "Theorem \label{thr:hairy_ball}".
% If the theorem is later changed to a lemma, so thr:hairy_ball is renamed lemma:hairy_ball, \thref{hairy_ball} will instead evaluate to "\Lemma \label{lemma:hairy_ball}".
% What happens if two labels with the same name (both thr:hairy_ball and lemma:hairy_ball) is undefined.
\makeatletter
\newcommand\thref[1]{%
	\@ifundefined{r@theorem:#1}{%
		\@ifundefined{r@lemma:#1}{%
			\@ifundefined{r@corollary:#1}{%
				Definition \ref {definition:#1}%
			}{%
				Corollary \ref{corollary:#1}%
			}%
		}{%
			Lemma \ref{lemma:#1}%
		}%
	}{%
		Theorem \ref{theorem:#1}%
	}%
}
\makeatother


\subtitle{\thesissubtitle}
\title{\thesistitle}

\setdepartment{\thesisdepartment}
\setcolor{dtured}



\begin{document}
\inserttitlepage


\begin{frame}{Outline} % 2 minutes
	\tableofcontents

\end{frame}
\note{

}


\section{Introduction} % 8 minutes
\begin{frame}{Motivation} % 3 minutes
    % Either a graph showing LLM progress or a thematic image
    % For a graph, the zwi blog post has a good one
\end{frame}
\note{
    LLMs getting powerful.

    Potential danger in several ways.

    Interpretability could be key to controlling and gaining confidence in LLMs.

    Neurons seem uninterpretable.

    Need another unit of interpretation.
}

\begin{frame}{Related work} % 3 minutes
    % Much work not published in academic journals.
    % Either on forums (Alignment forum) or in company blogs (OpenAI, Anthropic)
    % Circuits, automated interpretability, 
\end{frame}
\note{
    
}

\begin{frame}{Goals} % 2 minutes
    \begin{itemize}
        \item Confirm that SAE latents are more interpretable with novel interpretability metric (N2G)
        \item Investigate clusters in SAE latents based on N2G, density, and geometry
        \item Investigate the hypothesis that LMs exploit nearly orthogonal directions in MLP neurons
    \end{itemize}
\end{frame}
\note{
    The metrics we will cluster based on: interpretability (N2G), density, geometry    
}

\section{Background} % 8 minutes
% Possibly insert some kind of "preliminaries" slide
\begin{frame}{Neuron2Graph} % 4 minutes
    \only<1>{
    \begin{itemize}
        \item Automated interpretability method
        \item Models a feature by building a syntactical graph
    \end{itemize}
    }
    \only<2>{
        \begin{itemize}
            \item \texttt{"The"," big"," ten"}
        \end{itemize}
    }
    \only<3>{
        \begin{itemize}
            \item \texttt{"The"," last"," ten"}
        \end{itemize}
    }
    \includegraphics[height=0.7\textheight]{../images/gpt2-small_2_1817.pdf}
\end{frame}
\note{
    Is an automated interpretability method published last year by Alex Foote.
    Models a feature by building a syntactical graph based on a set of sample token strings.
    The point being that this model has an easily interpretable visualisation.
    Attempts to augment and prune these strings to find the simplest sufficient token patterns that activate the feature.
    
    Examples
}


\begin{frame}{Sparse Autoencoders} % 4 minutes
    \begin{align*}
        \vec y=&\mathrm{ReLU}\left(\mat W_e\vec x+\vec b\right)\\
        \hat{\vec x}=&\mat W_d\vec y\\
        \mathcal L(\vec x)=&\norm{\vec x-\hat{\vec x}}_2^2+\alpha\norm{\vec y}_1
    \end{align*}
    \centering
    \includegraphics[height=0.5\textheight]{../images/cunningham_sae_illustration.png}
\end{frame}
\note{
    Neurons seem to be uninterpretable.
    Linear representation hypothesis -> We need to find the directions in MLP neuron space that are interpretable.
    SAEs are a method of doing this.
    Describe structure (and loss) of SAE with image and equations.
    Describe how it fits into the model.
}


\section{Method} % 8 minutes
\begin{frame}{Models and dataset} % 1 minute
    
\end{frame}
\note{

}


\begin{frame}{Neuron2Graph} % 3 minutes
    
\end{frame}
\note{

}


\begin{frame}{Analysis} % 4 minutes
    
\end{frame}
\note{

}


\section{Results} % 9 minutes
\begin{frame}{Interpretability} % 3 minutes
    
\end{frame}
\note{

}


\begin{frame}{Density} % 3 minutes
    
\end{frame}
\note{

}


\begin{frame}{Geometry} % 3 minutes
    
\end{frame}
\note{

}


\section{Discussion} % 12 munutes
\begin{frame}{Results} % 5 minutes
    
\end{frame}
\note{

}


\begin{frame}{Methods} % 3 minutes
    
\end{frame}
\note{

}


\begin{frame}{Further Work} % 4 minutes

\end{frame}
\note{

}


\section{Conclusion} % 3 minutes
\begin{frame}{Conclusion} % 3 minutes
    
\end{frame}
\note{

}





\end{document}